\chapter{Validation}\label{validation}

The presented proposal was validated through two\,approaches.
The design and the usability of the interface was validated through
  solving several use cases of the target audience with the proposal.
Interest in the proposal by the target audience was validated by
  measuring the enthusiasm showed in the open source community.

\section{Plug-ins}\label{plugins}

More than fifteen\,plug-ins for \emph{Retext} were created to confirm if,
  and validate how, the proposals integrated together, and how the system
  worked.

The proposal solves the creation of natural language by default (use case
  \ref{list:use-case:2}), but these plug-ins solve several others.
The developed plug-ins included implementations for:

\begin{aenumerate}
\item Transforming so-called dumb punctuation marks into more
  typographically correct punctuation marks, solving use case
  \ref{list:use-case:4} \autocite*{wooorm/retext-smartypants-source-code};
\item Transforming emoji short-codes (:cat:) into real emoji
  \autocite*{wooorm/retext-emoji-source-code};
\item Detecting the direction of text
  \autocite*{wooorm/retext-directionality-source-code};
\item Detecting phonetics
  \autocite*{wooorm/retext-double-metaphone-source-code};
\item Detecting the stem of words
  \autocite*{wooorm/retext-porter-stemmer-source-code};
\item Finding grammatical units, solving use case
  \ref{list:use-case:5}  \autocite*{wooorm/retext-visit-source-code};
\item Finding text, even misspelled, solving use case
  \ref{list:use-case:7} \autocite*{wooorm/retext-search-source-code};
\item Detecting \gls{pos} tags \autocite*{wooorm/retext-pos-source-code};
\item Finding keywords and -phrases
  \autocite*{wooorm/retext-keywords-source-code};
\item Detecting the language of text, solving use case \ref{list:use-case:6}
  \autocite*{wooorm/retext-language-source-code}.
\item Detecting the sentiment of text, solving use case \ref{list:use-case:3}
  \autocite*{wooorm/retext-sentiment-source-code}.
\end{aenumerate}

\noindent These plug-ins listed and the other plug-ins solve all but one
  use cases of the target audience (§\,\ref{use-cases}).
The unsolved use case can be solved using the plug-in mechanism provided by
  \emph{Retext}.
Summarising natural language (use case \ref{list:use-case:1}) is not yet
  solved, but can be by implementing the stages mentioned in
  §\,\ref{automatic-summarisation} (p.\,\pageref{automatic-summarisation}).

During the development of these plug-ins, several problems were brought to
  light in the developed software.
These problems were recursively dealt with, back and forth, between the
  software and the plug-ins.
The software changed severely by these changes, which resulted in a
  better interface and usability.

An example of how the developed plug-ins changed the proposal, is the fact
  that the proposal initially did not provide information about punctuation
  in words.
Words could contain punctuation (such\,as ``I'm''), but these marks were not
  available to plug-ins.
Currently, the proposal allows for \emph{word} tokens to contain raw text
  tokens (``I'' and ``m'' in ``I'm'') and additional punctuation tokens (the
  apostrophe in ``I'm'').

\section{Reception}\label{reception}

To confirm interest by the target audience in the proposal, enthusiasm
  showed by the open source community was measured.
To initially spark interest, several websites and e-mail newsletters were
  contacted to feature \emph{Retext}, either in the form of an article or as
  a simple link.
This resulted in coverage on high-profile websites
  \autocite{dailyjs.com-natural-language-parsing-retext} and newsletters
  \autocites{nodeweekly.com-47}{javascriptweekly.com-193}
  {newspaper.io/javascript-2014-08-11}.
Later, organic growth resulted in features on link roundups
  \autocites{github.com-awesome-machine-learning}{github.com-awesome-nodejs},
  Reddit \autocites{reddit.com-mention-1}{reddit.com-mention-2}
  {reddit.com-mention-3}.

In turn, these publications resulted in positive reactions, such\,as on
  Twitter \autocites{twitter.com-mention-1}{twitter.com-mention-2}
  {twitter.com-mention-3}{twitter.com-mention-4}{twitter.com-mention-5}
  {twitter.com-mention-6}, other websites, and both feedback and fixes
  on GitHub \autocites{github.com-issue-1}{github.com-issue-2}
  {github.com-issue-3}{github.com-pull-request}.

Additionally, many of the target audience started \emph{following} the
  project on GitHub \autocite{github.com-stargazers}.
