%*******************************************************
% Addendum Production
%*******************************************************

\begingroup
\let\clearpage\relax
\let\cleardoublepage\relax
\let\cleardoublepage\relax

% work-around to have small caps also here in the headline
\manualmark
\markboth{\spacedlowsmallcaps{Production}}{\spacedlowsmallcaps{Production}}

\chapter*{Production}\label{addendum-production}
\addtocontents{toc}{\protect\vspace{\beforebibskip}}
\addcontentsline{toc}{chapter}{\tocEntry{Production}}

As described in Validation: Plug-ins (§\,\ref{plugins},
  p.\,\pageref{plugins}), all use cases can be implemented with \emph{Retext}.
Additionally, other challenges of the target audience can also be
  implemented.

Before I began work on a proposal, I wrote pseudocode to solve several use
  cases which depended on the (non-existent) proposal.
This defined what functionality the proposal needed to expose to solve the use
  cases.
I created the unit tests for, and started development of, \gls{textom}.
\gls{textom} is the object model which exposes functionality to analyse and
  manipulate natural language (§\,\ref{object-model},
  p.\,\pageref{object-model}).

During the development of both \gls{textom} and the pseudocode, missing
  functionality was added and useless functionality was removed.
An example of the former was the addition of \emph{events}, a mechanism to
  detect changes.
An example of useless functionality was `Range', a mechanism to store a
  sequence of content within a TextOM tree, which was not needed for
  most pseudocode examples.

As described in Design \& Architecture: Object Model (§\,\ref{object-model},
  p.\,\pageref{object-model}), \gls{textom} was designed in similarity to the
  \gls{dom}.
Multiple concepts were taken from the \gls{dom} and used in \gls{textom},
  such as events, errors, children and parents, and types.
Additionally, some concepts were taken from the initial implementation,
  \emph{MicroType.js}, such as functionality to split a node.

When the functionality to solve the use cases was implemented in \gls{textom},
  I began work on the parser (§\,\ref{parser-parse-latin},
  p.\,\pageref{parser-parse-latin}), to create the object model from
  English input.
This parser, \emph{parse-latin}, tokenised words, punctuation, white space,
  sentences, and paragraphs, and transformed these tokens into \gls{textom}
  objects.
The addition of a parser to create a \gls{textom} tree from arbitrary
  English, made it possible to start testing \gls{textom} itself and the use
  case code (formerly pseudocode).
The main problem brought to light during the development of the parser in
  combination with the use case code, was the incorrect
  handling of inner-word punctuation symbols.
This conundrum is described in the last paragraph of Plug-ins
  (§\,\ref{plugins}, p.\,\pageref{plugins}).

As described in Design \& Architecture: Parser (§\,\ref{parser-parse-latin},
  p.\,\pageref{parser-parse-latin}), \emph{parse-latin} tokenises sentences
  similar to \emph{Emphasis}, and inner-sentence content similar
  to \emph{Treebank}.
However, there are many differences between these implementations and
  \emph{parse-latin}.
\emph{Emphasis} does not handle sentences ending in exclamation
  points and interrogative points, or affix punctuation, and \emph{Treebank}
  works only for English text and discards everything non-word.
\emph{parse-latin} handles these problems as expected.

The last major part of the proposal, the system that binds everything
  together (as described in §\,\ref{natural-language-system-retext},
  p.\,\pageref{natural-language-system-retext})), is \emph{Retext}.
\emph{Retext} makes it easy for developers to work with both \gls{textom}
  and \emph{parse-latin}.
Additionally, \emph{Retext} provides a plug-in system so users can create
  plug-ins for others to use, and in turn enables them to use others'
  plug-ins inside their projects.

I tested how if system integrated together, and if the plug-in interface
  worked, by rewriting the use case code into \emph{Retext} plug-ins.
The major problem brought to light was the fact that other parser might be
  required, instead of \emph{parse-latin}.
This proved problematic because \emph{Retext} itself did not know about
  \gls{textom}: \emph{parse-latin} directly used \gls{textom}.
To enable the use of other parsers (the later created \emph{parse-english}
  and \emph{parse-dutch}, or parsers for other scripts),
  \emph{Retext} and \emph{parse-latin} changed severely.

How much work went into creating the proposal is best explained by looking
  at several metrics provided by both Git and GitHub.
The metrics looked at are the number of \emph{commits}---a term for a
  significant change, such as updating a version, fixing a typographic
  mistake, or large rewrites---and lines added or removed.
Note that the latter, line changes, counts a changed line (such as the
  addition of a single missing letter) as both a removed, and an added line.
In addition, note that the following metrics cover the complete code base,
  including documentation, benchmarks, licenses, unit tests, and more.

\emph{Retext}, \emph{parse-latin}, \emph{parse-english}, \emph{parse-dutch},
  and \gls{textom} together were created in 759 commits, 230,319 added lines,
  and 193,804 removed lines.
All 17 plug-ins used to validate the proposal\footnote{\emph{retext-ast},
    \emph{retext-content},
    \emph{retext-directionality}, \emph{retext-dom},
    \emph{retext-double-metaphone}, \emph{retext-emoji},
    \emph{retext-keywords}, \emph{retext-language}
    \emph{retext-link}, \emph{retext-metaphone},
    \emph{retext-porter-stemmer}, \emph{retext-pos}, \emph{retext-range},
    \emph{retext-search}, \emph{retext-sentiment}, \emph{retext-smartypants},
    and \emph{retext-visit}.},
  were created in 370 commits, 115,732 added lines, and 101,212 removed lines.

Note that does not cover other work that went into creating the proposal,
  such as demos.
Additionally, I wrote many of the algorithms use by the plug-ins seperatly
  from \emph{Retext}, for others to use\footnote{\emph{gemoji},
    \emph{direction}, \emph{franc}, \emph{stemmer}, \emph{metaphone},
    \emph{double-metaphone}, \emph{dice-coefficient},
    \emph{lancaster-stemmer}, \emph{levenshtein-edit-distance},
    \emph{polarity}, \emph{afinn-111}, \emph{weasels}.
  }.

\endgroup
