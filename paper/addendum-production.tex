%*******************************************************
% Addendum Production
%*******************************************************

\begingroup
\let\clearpage\relax
\let\cleardoublepage\relax
\let\cleardoublepage\relax

% work-around to have small caps also here in the headline
\manualmark
\markboth{\spacedlowsmallcaps{Production}}{\spacedlowsmallcaps{Production}}

\chapter*{Production}\label{addendum-production}
\addtocontents{toc}{\protect\vspace{\beforebibskip}}
\addcontentsline{toc}{chapter}{\tocEntry{Production}}

Retext is capable of solving all of the target audience's use cases through
  plug-ins, as described in Plug-ins (§\,\ref{plugins},
  p.\,\pageref{plugins}).

Before I started the development of the proposal, I wrote pseudocode to
  solve several use cases which depended on the (non-existent) proposal.
The pseudocode defined what functionality the proposal should expose.

Then, I wrote the unit tests for the required functionality, and started
  the development of \gls{textom} (§\,\ref{object-model},
  p.\,\pageref{object-model}).
During the development of both \gls{textom} and the pseudocode, missing
  functionality was added and useless functionality was removed.
An example of the former is the addition of events, a mechanism to detect
  changes.
An example of the latter is the removal of `Range', a mechanism to store a
  sequence of content within a \gls{textom} tree.

As described in Object Model (§\,\ref{object-model},
  p.\,\pageref{object-model}), \gls{textom} was designed in similarity to the
  \gls{dom}.
Multiple concepts were inspired by the \gls{dom}, such as events, errors,
  children and parents, and types.
Additionally, some concepts were inspired by the initial implementation,
  \emph{MicroType.js}, such as functionality to split a node.

When the functionality to solve the use cases was exposed by \gls{textom},
  work started on the parser (§\,\ref{parser-parse-latin},
  p.\,\pageref{parser-parse-latin}).
The parser, \emph{parse-latin}, tokenised arbitrary Latin script text into
  words, punctuation, white space, sentences, and paragraphs, and transformed
  these tokens into a \gls{textom} tree.
The parser enabled the validation of \gls{textom} itself, which in turn
  resulted in the transformation of the pseudocode into code.
Among other things, this initial testing phase revealed the incorrect
  handling of inner-word punctuation symbols.
The last paragraph of Plug-ins (§\,\ref{plugins}, p.\,\pageref{plugins})
  describes this conundrum.

As described in Parser (§\,\ref{parser-parse-latin},
  p.\,\pageref{parser-parse-latin}), \emph{parse-latin} tokenises sentences
  similar to \emph{Emphasis}, and inner-sentence content similar to
  \emph{Treebank}.
However, \emph{parse-latin} contains added functionality to tokenise smarter.
For example, \emph{Emphasis} ignores sentences ending in exclamation- or
  interrogative points, and ignores affix punctuation, and \emph{Treebank}
  discards everything non-word, while \emph{parse-latin} handles these
  problems as expected.

After \gls{textom} and \emph{parse-latin}, work started on the final
  part of the proposal, \emph{Retext}: the system that binds everything
  together (as described in §\,\ref{natural-language-system-retext},
  p.\,\pageref{natural-language-system-retext}).
In addition to providing \gls{textom} and \emph{parse-latin}, \emph{Retext}
  provided a plug-in system enabling users to develop plug-ins for others
  to implement, and in turn enables users to implement others' plug-ins
  in projects.

By rewriting the use case code as \emph{Retext} plug-ins, I validated if
  the system integrated together and if the plug-in interface functioned
  properly.
This validation phase revealed a major problem: to function, \emph{Retext}
  required \emph{parse-latin}.
However, other parsers were needed for other scripts and other languages.
To enable the use of other parsers, both \emph{Retext}'s and
  \emph{parse-latin}'s code changed significantly.
To validate if and how this change functioned, I developed the
  \emph{parse-english} and \emph{parse-dutch}.

\medskip\noindent Looking at several metrics provided by both Git and
  GitHub best explains the amount of effort that went into the development
  of the proposal.
The provided metrics of importance are both added and removed lines, and
  significant changes (called \emph{commits} in Git), such as version
  updates, grammar fixes, and complete rewrites.
Note that the former includes small changes, such as the addition of a single
  missing letter, as both an added and a removed line.
Additionally, note that the metrics cover the complete code base, including
  documentation, benchmarks, licenses, unit tests, and more.

The development of \emph{Retext}, \emph{parse-latin}, \emph{parse-english},
  \emph{parse-dutch}, and \gls{textom} together consisted of 759 commits,
  230,319 added lines, and 193,804 removed lines.
The development of all seventeen plug-ins to validate the
  proposal\footnote{The seventeen plug-ins are \emph{retext-ast},
    \emph{retext-content},
    \emph{retext-directionality}, \emph{retext-dom},
    \emph{retext-double-metaphone}, \emph{retext-emoji},
    \emph{retext-keywords}, \emph{retext-language}
    \emph{retext-link}, \emph{retext-metaphone},
    \emph{retext-porter-stemmer}, \emph{retext-pos}, \emph{retext-range},
    \emph{retext-search}, \emph{retext-sentiment}, \emph{retext-smartypants},
    and \emph{retext-visit}.},
  consisted of 370 commits, 115,732 added lines, and 101,212 removed lines.

Note that these metrics do not completely cover the amount of effort that
  went into the development of the proposal:

\begin{enumerate}
\item The metrics cover functioning code, not pseudocode or the work that
  went into its creation.
\item The metrics exclude the creation of demos for eleven plug-ins.
\item The metrics exclude several algorithms behind the plug-ins,
  which were published separately\footnote{Algorithms inlcude
    \emph{gemoji}, \emph{direction}, \emph{franc}, \emph{stemmer},
    \emph{metaphone}, \emph{double-metaphone}, \emph{dice-coefficient},
    \emph{lancaster-stemmer}, \emph{levenshtein-edit-distance},
    \emph{polarity}, and \emph{afinn-111}.
  }.
\end{enumerate}

\endgroup
