\chapter{Conclusion}\label{conclusion}

This chapter\,consists of a\,short summary (§\,\ref{summary}), a\,list of
  limitations and suggestions for future work
  (§\,\ref{limitations-future-work}), and a\,list of conclusions
  (§\,\ref{conclusions}).

\section{Summary}\label{summary}

\gls{nlp} covers many challenges.
The process of accomplishing these challenges touches on well-defined stages.
Such\,as \emph{tokenisation}, the focus of this paper.
Current implementations on the web platform are lacking.
In part, because techniques such\,as \emph{supervised learning} do not work
  on the web.

The audience that benefits the most from better parsing on the web platform,
  are web developers, a\,group which is more interested in practical use, and
  less so in theoretical applications.

The presented proposal is split up in several solutions: a\,specification,
  a\,parser, and an\,object model.
These solutions come together in a\,proposal: \emph{Retext}, a\,complete
  natural language system.
\emph{Retext} takes care of parsing natural language and enables users to
  create and use plug-ins.

The proposal was validated both by solving the audience's use cases with
  \emph{Retext}, and by measuring the audience's enthusiasm for \emph{Retext}.

\section{Limitations \& Future Work}\label{limitations-future-work}

The proposal leaves open many areas of interest for future investigation.
Some of which are featured here.

\begin{aenumerate}
\item\emph{Internationalisation}\,---\,Currently, the proposal is only
    tested on Latin script languages.
  The software was developed with other languages and scripts, such\,as
    Arabic, Hangul, Hebrew, and Kanji, in mind.
  Future work could expand support to include these scripts;
\item\emph{Difference Application}\,---\,Currently, the proposal does not
    support difference application.
  When a\,word is added at the end of a\,sentence, all steps to produce the
    output have to be revisited.
  Although the proposal is created with this in mind, no support has been
    added.
  Future work could include difference application support;
\item\emph{Non-rule Based Parsing}\,---\,Currently, \gls{nlcst} trees are
    created with rule based parsers.
  But, corpora based parsers could also produce these trees.
  Future work could investigate and implement such supervised learning
    approaches.
\item\emph{Academic Goals}\,---\,Currently, the proposals cater to practical use
    cases.
  Future work could expand on this purview and implement more academic
    goals.
\item\emph{Semantic Units}\,---\,Currently, the proposals provide syntactic
    units.
  Future work could expand on this by providing information about phrases
    and clauses to users.
\item\emph{Source Formats}\,---\,Currently, the parsers each require plain
    text input.
  Future work could expand on this by allowing other input formats, such\,as
    MarkDown or \TeX.
\item\emph{Heighten Performance}\,---\,The decision made to adopt
    an\,object-oriented approach for analysation and manipulation, came at
    a\,huge performance cut.
  When implementing both \gls{textom} and \emph{parse-latin} over just
    \emph{parse-latin}, performance decreases over 90\%.
  Future work should investigate and implement better performance.
\end{aenumerate}

\section{Conclusions}\label{conclusions}

This section evaluates if the research question and sub-questions are
  answered, and if the research objective is reached.

\subsection{Current Possibilities \&
  Deficiencies}\label{q-current-possibilities}

In this subsection, research sub-question \ref{list:research-question-context}
  is evaluated (§\,\ref{research-sub-questions},
  p.\,\pageref{list:research-question-implementation}).

\begin{quote}
  \textit{What current implementations exist? What does not yet exist?}
\end{quote}

\noindent\gls{nlp} covers many challenges.
Within the scope of this thesis, only \emph{tokenisation} was covered
  (§\,\ref{scope}, p.\,\pageref{scope}).
Most current implementations use (lacking) tokenisation as part of a\,larger
  challenge (§\,\ref{challenges}, p.\,\pageref{challenges}).
Implementations that provide tokenisation capabilities to other tasks,
  are lacking (§\,\ref{stages}, p.\,\pageref{stages}).
It is concluded that a\,quality implementations that offers tokenisation
  within the scope, does not yet exist.

\subsection{Quality Implementation}\label{q-quality-implementation}

In this subsection, research sub-question
  \ref{list:research-question-implementation} is evaluated
  (§\,\ref{research-sub-questions},
  p.\,\pageref{list:research-question-implementation}).

\noindent\begin{quote}
  \textit{What makes a\,good \gls{api} design? What makes a\,good
    implementation?}
\end{quote}

The proposal should meet several requirements, other than the use cases,
  to better suit the wishes of the target audience.
This includes open source development and easy installation, readable code,
  tested results, high performance, and a\,good interface design.
Concluded was that by following these best practises for code and creating
  an\,interface similar to for the target audience familiar projects,
  a\,\emph{good} implementation can be created (§\,\ref{requirements},
  p.\,\pageref{requirements}).

\subsection{The Target Audience's Use Cases}\label{q-use-cases}

In this subsection, research sub-question
  \ref{list:research-question-audience} is evaluated
  (§\,\ref{research-sub-questions},
  p.\,\pageref{list:research-question-audience}).

\noindent\begin{quote}
  \textit{What would they use an\,implementation for? What would they
    \emph{not} use the implementation for?}
\end{quote}

The audience that benefits the most from the proposal, are web developers
  (§\,\ref{target-audience}, p.\,\pageref{target-audience}).
Not every challenge in the field is of interest to the web developer.
More academic areas of \gls{nlp}, do not fit well with the goals of web
  developers (§\,\ref{use-cases}, p.\,\pageref{use-cases}).
Research for this paper found that the target audience would use the
  implementation for several use cases.

\subsection{Research Question}\label{q-research-question}

In this section, the defined research question is evaluated
  (§\,\ref{research-question}, p.\,\pageref{research-question}).

\noindent\begin{quote}
  \textit{How can a\,specification and program, that exposes an\,\acrshort{api}
    for text manipulation, based on use cases of developers on the web,
    be developed?
  }
\end{quote}

\noindent\emph{Current possibilities and deficiencies}
  (§\,\ref{q-current-possibilities}, p.\,\pageref{q-current-possibilities})
  concludes that quality implementations do not exist.
\emph{Quality implementation} (§\,\ref{q-quality-implementation},
  p.\,\pageref{q-quality-implementation}) concludes that a quality
  implementation can be created by followed several guidelines.
\emph{The target audience's use cases} (§\,\ref{q-use-cases},
  p.\,\pageref{q-use-cases}) concludes that the target audience would use the
  implementation for several use cases.

The answers to the sub-questions, answer the complete research question,
  within the scope (§\,\ref{scope}, p.\,\pageref{scope}).

\subsection{Research Objective}\label{q-research-objective}

\emph{How} to create a\,specification and program was answered by the research
  question.
The working proposal reaches the research objective.
This was validated by the more than fifteen\,plug-ins solving the target
  audience's use cases (§\,\ref{plugins}, p.\,\pageref{plugins}).

In addition to \emph{reaching} this objective, the measured enthusiasm
  showed by the target audience for the proposal confirmed the interest in
  the proposal (§\,\ref{reception}, p.\,\pageref{reception}).
