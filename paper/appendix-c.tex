%********************************************************************
% Appendix C
%*******************************************************

\chapter{Appendix C: DOM}\label{appendix-c-dom}

The \gls{dom} specification defines a platform-neutral model for errors,
events, and (for this paper, the primary feature) node trees. XML-based
documents can be represented by the \gls{dom}.

Consider the following HTML document:

\begin{lstlisting}[language=HTML]
<!DOCTYPE html>
<html class=e>
    <head><title>Aliens?</title></head>
    <body>Why yes.</body>
</html>
\end{lstlisting}

Is represented by the \gls{dom} as follows:

\begin{lstlisting}
|- Document
   |- Doctype: html
   |- Element: html class="e"
      |- Element: head
      |  |- Element: title
      |     |- Text: Aliens?
      |- Text: \n␣
      |- Element: body
        |- Text: Why yes.\n
\end{lstlisting}

The \gls{dom} interfaces of bygone times were widely considered horrible,
but newer features seem to be gaining popularity in the web authoring
community as broader implementation across user agents is reached.
