%*******************************************************
% Executive Summary
%*******************************************************
% to have the executive summary a bit from the rest in the toc
\refstepcounter{dummy}
\addtocontents{toc}{\protect\vspace{\beforebibskip}}
\addcontentsline{toc}{chapter}{\tocEntry{Executive Summary}}

\begingroup
\let\clearpage\relax
\let\cleardoublepage\relax
\let\cleardoublepage\relax

\chapter*{Executive Summary}

\gls{nlp} covers many tasks, but the process of accomplishing
  these goals touches on well-defined stages
  (§\,\ref{natural-language-processing},
  p.\,\pageref{natural-language-processing}).
Such as \emph{tokenisation}, the focus of the proposal.
Current implementations on the web platform are lacking
  (§\,\ref{implementations}, p.\,\pageref{implementations}).
In part, because advanced machine learning techniques (such as
  \emph{supervised learning}) do not work on the web
  (§\,\ref{using-corpora-for}, p.\,\pageref{using-corpora-for}).

The audience that benefits the most from better parsing on the web platform,
  are web developers, a group which is more interested in practical use, and
  less so in theoretical applications
  (§\,\ref{target-audience}, p.\,\pageref{target-audience}).

The target audience's use cases for \gls{nlp} on the web are vast. Examples
  include automatic summarisation, sentiment recognition, spam detection,
  typographic enhancements, counting words, language recognition, and more
  (§\,\ref{use-cases}, p.\,\pageref{use-cases}).

The presented proposal is split into several smaller solutions.
These solutions come together in a proposal: \emph{Retext}, a complete
  natural language system (§\,\ref{design}, p.\,\pageref{design}).
\emph{Retext} takes care of parsing natural language and enables users to
  create and use plug-ins (§\,\ref{natural-language-system-retext},
  p.\,\pageref{natural-language-system-retext}).

Parsing is delegated to \emph{parse-latin} and others, which first\,tokenise
  text into a list of words, punctuation, and white space. Later, these
  tokens are parsed into a syntax tree, containing paragraphs, sentences,
  embedded content, and more. Their intellect extends several well known
  techniques (§\,\ref{parser-parse-latin}, p.\,\pageref{parser-parse-latin}).

The objects returned by \emph{parse-latin} and others are defined by
  \acrshort{nlcst}. \acrshort{nlcst} defines the syntax for these objects.
  \acrshort{nlcst} is designed in similarity to other popular syntax tree
  specifications (§\,\ref{syntax}, p.\,\pageref{syntax}).

The interface to analyse and manipulate these object is implemented by
  \gls{textom}. \gls{textom} is created in similarity to other, for the target
  audience well known, techniques
  (§\,\ref{object-model}, p.\,\pageref{object-model}).

The proposal was validated both by solving the audience's use cases with
  \emph{Retext}, and by measuring the audience's enthusiasm for \emph{Retext}.
Use cases were validated by implementing many as plug-ins for
  \emph{Retext} (§\,\ref{plugins}, p.\,\pageref{plugins}).
The enthusiasm showed by the target audience on social networks, through
  e-mail, and social coding was positive
  (§\,\ref{reception}, p.\,\pageref{reception}).

\endgroup
