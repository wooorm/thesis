%*******************************************************
% Addendum Use Cases
%*******************************************************

\begingroup
\let\clearpage\relax
\let\cleardoublepage\relax
\let\cleardoublepage\relax

% work-around to have small caps also here in the headline
\manualmark
\markboth{\spacedlowsmallcaps{Validation}}{\spacedlowsmallcaps{Validation}}

\chapter*{Validation}\label{addendum-validation}
\addtocontents{toc}{\protect\vspace{\beforebibskip}}
\addcontentsline{toc}{chapter}{\tocEntry{Validation}}

Most physical and virtual products work ``well'' if the target audience can
  use the product without needing to resort to a manual.
Code is, however, different: it is both expected and normal for a user of
  an \gls{api} to resort to the manual, the documentation.

User testing the proposal in this thesis, an \gls{api}, would validate the
  documentation instead of the interface.
Useable documentation is certainly of importance, but this thesis does
  not propose documentation.

Therefore, I chose to validating the target audience's reception.
As mentioned in Reception (§\,\ref{reception}, p.\,\pageref{reception}),
  I first contacted high-profile web sites, including Daily\textsc{js}.
Additionally, I posted on Echo\textsc{js} and Reddit.
This resulted in mentions on newsletters and link roundups.
In turn, this resulted in many of the target audience mentioning \emph{Retext}
  and other projects in the \emph{Retext}-family on social networks such as
  Twitter and Google+.

A few of these mentions are listen below, either because they were re-shared
  numerous times, or contain unique understanding of, or use cases for, the
  proposal.

\begin{quote}
  \textit{This. Just. Made. My. Brain. Explode.
    \href{https://github.com/wooorm/retext}{\nolinkurl{github.com/wooorm/retext}}
  }

  \medskip ---\,Trent\,Oswald\,(therebelrobot). 1 Aug 2014, 08:15 a.m.
  \href{https://twitter.com/therebelrobot/status/495226217805524992}{Tweet}.
\end{quote}

\hrule

\begin{quote}
  \textit{Retext is a JavaScript natural language parser useful for doing
    things like removing profanity, analyzing text, etc.
    \href{https://github.com/wooorm/retext}{\nolinkurl{github.com/wooorm/retext}}
  }

  \medskip ---\,Brian\,Rinaldi\,(remotesynth). 1 Aug 2014, 11:15 a.m.
  \href{https://twitter.com/remotesynth/status/495271619737423872}{Tweet}.
\end{quote}

\hrule

\begin{quote}
  \textit{Convert text-strings to TextOM for easy manipulations
    \href{https://github.com/wooorm/retext}{\nolinkurl{github.com/wooorm/retext}}
    \#github \#npm \#javascript \#nodejs
  }

  \medskip ---\,Thomas\,Strobl\,(tom2strobl). 15 Aug 2014, 06:00 a.m.
  \href{https://twitter.com/tom2strobl/status/500265708866256896}{Tweet}.
\end{quote}

\hrule

\begin{quote}
  \textit{I'm finding retext a useful Natural Language Processing
    framework when working with \#IBMWatson
    \href{https://github.com/wooorm/retext}{\nolinkurl{github.com/wooorm/retext}}
  }

  \medskip ---\,Stephen\,Keep\,(stephenkeep). 12 Aug 2014, 02:46 a.m.
  \href{https://twitter.com/stephenkeep/status/499129742407532545}{Tweet}.
\end{quote}

\hrule

\begin{quote}
  \textit{Retext: Extensible System for Analysing and Manipulating Natural
    Language
    \href{https://github.com/wooorm/retext}{\nolinkurl{github.com/wooorm/retext}}
  }

  \medskip ---\,JavaScript Daily\,(JavaScriptDaily). 10 Aug 2014, 01:41 p.m.
  \href{https://twitter.com/JavaScriptDaily/status/498569727854522368}{Tweet}.
\end{quote}

\hrule

\begin{quote}
  \textit{English (latin) language parser in JavaScript:
    \href{https://github.com/wooorm/parse-english}{\nolinkurl{github.com/wooorm/parse-english}}
    - aka, convert English text into an AST for better + easier processing.
    Lots of fun possibilities!
    Interactive demo (latin parser):
    \href{http://wooorm.github.io/parse-latin/}{\nolinkurl{wooorm.github.io/parse-latin/}}
  }

  \medskip ---\,Ilya\,Grigorik\,(IlyaGrigorik). 8 Aug 2014.
  \href{https://plus.google.com/+IlyaGrigorik/posts/RR75ZLceDHU}{Google+}.
\end{quote}

\noindent Many more can be found by searching Twitter for tweets containing
  either my GitHub handle (wooorm,
  \href{https://twitter.com/search?f=realtime&q=wooorm}{\nolinkurl{twitter.com/search?q=wooorm}})
  or the GitHub project name (\emph{Retext},
  \href{https://twitter.com/search?f=realtime&q=retext}{\nolinkurl{twitter.com/search?q=retext}}).

In addition to mentioning the proposal on social networks many of the target
  audience started \emph{following} the proposal.
At time of writing this addendum, 22 August, 2014, 451 of the target
  audience followed \emph{Retext}, including members of the target audience
  working at companies such as Google, \gls{ibm}, Facebook, Adobe, Mozilla,
  Vimeo, Yandex, Spotify, Telegraaf Media Groep, and Turner Broadcasting.

The fact that \emph{Retext} was classified as \emph{trending} by GitHub (a
  list of projects receiving the most stars) multiple times for projects in
  \gls{ecmascript}, gives context to this number.
Such as receiving third\,place on both August 1 and 2, 2014.

These followers (named \emph{stargazers} on GitHub) do not necessarily convert
  to actual users, they merely convey interest.
Actual usage is harder to detect.

One way to detect actual use is by looking at download rates, such as
  those displayed by \gls{npm}, the main package manager used to download
  and use the proposal.
According to \gls{npm}, between 22 July 2014 and 22 August 2014, \emph{Retext}
  was downloaded 496 times\footnote{
    \href{https://nodei.co/npm/retext.png?downloads=true}{\nolinkurl{nodei.co/npm/retext.png?downloads=true}}.
  }, \emph{parse-latin} 529 times\footnote{
    \href{https://nodei.co/npm/parse-latin.png?downloads=true}{\nolinkurl{nodei.co/npm/parse-latin.png?downloads=true}}.
  }, \emph{parse-english} 381 times\footnote{
    \href{https://nodei.co/npm/parse-english.png?downloads=true}{\nolinkurl{nodei.co/npm/parse-english.png?downloads=true}}.
  },
  and \gls{textom} 460 times\footnote{
    \href{https://nodei.co/npm/textom.png?downloads=true}{\nolinkurl{nodei.co/npm/textom.png?downloads=true}}.
  }.

Another way to detect actual use is by looking at by the target audience
  raised issues and proposed fixes, mentioned in Reception (§\,\ref{reception},
  p.\,\pageref{reception}).

Note that the number of proposed fixes and raised issues, as well as the
  download rates, are still not very high.
My current working theory for this is that the target audience is not
  regularly concerned with the topic, \gls{nlp}, and the focus,
  \emph{tokenisation}, of the project.
The project is too new to currently expect the target audience to use it
  regularly.

The future holds whether or not this working theory is correct.
However, the mentions on social networks, newsletters and web sites,
  the feature requests and bug reports, and the numerous bookmarks on
  GitHub and downloads on \gls{npm}, do certainly convey interest.

\endgroup
