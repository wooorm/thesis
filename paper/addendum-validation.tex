%*******************************************************
% Addendum Use Cases
%*******************************************************

\begingroup
\let\clearpage\relax
\let\cleardoublepage\relax
\let\cleardoublepage\relax

% work-around to have small caps also here in the headline
\manualmark
\markboth{\spacedlowsmallcaps{Validation}}{\spacedlowsmallcaps{Validation}}

\chapter*{Validation}\label{addendum-validation}
\addtocontents{toc}{\protect\vspace{\beforebibskip}}
\addcontentsline{toc}{chapter}{\tocEntry{Validation}}

Most physical and virtual products work ``well'', if the target audience can
  use the product without needing to resort to a manual.
Code is, however, different.
It is both expected and normal for a user of an \gls{api} to resort
  to the manual, the documentation.

Useable documentation is certainly of high importance, but this thesis does
  not propose documentation.
Therefore, user testing the proposal in this thesis, an \gls{api}, would
  validate the documentation, rather than the interface.

Instead I opted for validating the target audience's reception.
As mentioned in Reception (§\,\ref{reception}, p.\,\pageref{reception}),
  I first contacted some high-profile web sites, including Daily\textsc{js}.
Additionally, I posted on Echo \textsc{js} and Reddit.
This later resulted in mentions on newsletters and link roundups.
In turn this resulted in many peers mentioning \emph{Retext} and other
  projects in the \emph{Retext}-family on social networks such as Twitter
  and Google+.

A few of these mentions are listen below, either because they were shared
  numerous times, or contain unique understanding of, or use cases for, the
  projects.

\begin{quote}
  \textit{This. Just. Made. My. Brain. Explode.
    \href{https://github.com/wooorm/retext}{\nolinkurl{github.com/wooorm/retext}}
  }

  \medskip ---\,Trent\,Oswald\,(therebelrobot). 1 Aug 2014, 08:15 a.m.
  \href{https://twitter.com/therebelrobot/status/495226217805524992}{Tweet}.
\end{quote}

\begin{quote}
  \textit{Retext is a JavaScript natural language parser useful for doing
    things like removing profanity, analyzing text, etc.
    \href{https://github.com/wooorm/retext}{\nolinkurl{github.com/wooorm/retext}}
  }

  \medskip ---\,Brian\,Rinaldi\,(remotesynth). 1 Aug 2014, 11:15 a.m.
  \href{https://twitter.com/remotesynth/status/495271619737423872}{Tweet}.
\end{quote}

\begin{quote}
  \textit{Convert text-strings to TextOM for easy manipulations
    \href{https://github.com/wooorm/retext}{\nolinkurl{github.com/wooorm/retext}}
    \#github \#npm \#javascript \#nodejs
  }

  \medskip ---\,Thomas\,Strobl\,(tom2strobl). 15 Aug 2014, 06:00 a.m.
  \href{https://twitter.com/tom2strobl/status/500265708866256896}{Tweet}.
\end{quote}

\begin{quote}
  \textit{I'm finding retext a useful Natural Language Processing
    framework when working with \#IBMWatson
    \href{https://github.com/wooorm/retext}{\nolinkurl{github.com/wooorm/retext}}
  }

  \medskip ---\,Stephen\,Keep\,(stephenkeep). 12 Aug 2014, 02:46 a.m.
  \href{https://twitter.com/stephenkeep/status/499129742407532545}{Tweet}.
\end{quote}

\begin{quote}
  \textit{Retext: Extensible System for Analysing and Manipulating Natural
    Language
    \href{https://github.com/wooorm/retext}{\nolinkurl{github.com/wooorm/retext}}
  }

  \medskip ---\,JavaScript Daily\,(JavaScriptDaily). 10 Aug 2014, 01:41 p.m.
  \href{https://twitter.com/JavaScriptDaily/status/498569727854522368}{Tweet}.
\end{quote}

\begin{quote}
  \textit{English (latin) language parser in JavaScript:
    \href{https://github.com/wooorm/parse-english}{\nolinkurl{github.com/wooorm/parse-english}}
    - aka, convert English text into an AST for better + easier processing.
    Lots of fun possibilities!
    Interactive demo (latin parser):
    \href{http://wooorm.github.io/parse-latin/}{\nolinkurl{wooorm.github.io/parse-latin/}}
  }

  \medskip ---\,Ilya\,Grigorik\,(IlyaGrigorik). 8 Aug 2014.
  \href{https://plus.google.com/+IlyaGrigorik/posts/RR75ZLceDHU}{Google+}.
\end{quote}

\noindent Many more can be found by searching twitter for tweets containing either
  my GitHub handle (wooorm,
  \href{https://twitter.com/search?f=realtime&q=wooorm}{\nolinkurl{twitter.com/search?q=wooorm}})
  or the GitHub project name (\emph{Retext},
  \href{https://twitter.com/search?f=realtime&q=retext}{\nolinkurl{twitter.com/search?q=retext}}).

In addition to mentioning the project on social networks many of the target
  audience started \emph{following} the project.
At time of writing this addendum, 22 august, 2014, 451 of the target
  audience follow \emph{Retext}.
This includes developers working at Google, \gls{ibm}, Facebook, Adobe,
  Mozilla, Vimeo, Yandex, Spotify, Telegraaf Media Groep, Turner Broadcasting,
  and many more.

\endgroup
