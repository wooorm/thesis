\chapter{Research Framework}\label{research-framework}

This chapter states the objective of this paper (§\,\ref{research-objective}).
From the objective a research question is drafted
  (§\,\ref{research-question}).
Additionally, from the research question, several research questions are
  drafted, acting as a guideline for the proposal
  (§\,\ref{research-questions}).

\section{Research Objective}\label{research-objective}

Using the context described in chapter\,\ref{context} (p.\,\pageref{context}),
  the objective of this research project is constructed:

\begin{quote}
  \textit{A generic documentation (the ``specification'') and example
    implementation (the ``program'') that exposes an interface (the
    ``\acrshort{api}'') for the topic (``text manipulation'') based on real
    use cases of potential users (the ``developer'') on the platform (the
    ``web'').
  }
\end{quote}

\section{Research Question}\label{research-question}

This research objective leads to a research question:

\begin{quote}
  \textit{How can a specification and program, that exposes an \acrshort{api}
    for text manipulation, based on use cases of developers on the web,
    be developed?
  }
\end{quote}

\section{Research Questions}\label{research-questions}

The previously defined research question is split into several sub-questions.
These form the basis and a guide to answer the research question, to reach
  the research objective.

\begin{enumerate}
\item
  What are current possibilities and deficiencies in \gls{nlp}?
  \begin{enumerate}
    \item What current implementations exist?
    \item What does not yet exist?
  \end{enumerate}
\item
  How to ensure a quality implementation for the target audience?
  \begin{enumerate}
    \item What makes a good \gls{api} design?
    \item What makes a good implementation?
  \end{enumerate}
\item
  What are the target audience's use cases for an implementation?
  \begin{enumerate}
    \item What would they use an implementation for?
    \item What would they \emph{not} use the implementation for?
  \end{enumerate}
\end{enumerate}
