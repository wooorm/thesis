\chapter{Design \& Architecture}\label{design}

The in this paper presented solution to the problem of \gls{nlp} on the
client side is split up in multiple small proposals. Each sub-proposal
solves a sub-problem.

\begin{aenumerate}
\item\acrshort{nlcst} defines a standard for classifying grammatical units
  understandable for machines;
\item\emph{parse-latin} classifies natural language according to
  \acrshort{nlcst};
\item\gls{textom} provides an interface for analysing and manipulating output
  provided by \emph{parse-latin};
\item\emph{Retext} provides an interface for transforming natural language
  into an object model and exposes an interface for plug-ins.
\end{aenumerate}

\noindent The decoupled approach taken by the provided solution enables other
  developers to implement their own software to replace sub-proposals.
For example, other parties could create a parser for the Chinese language and
  use it instead of \emph{parse-latin} to classify natural language according
  to \gls{nlcst}, or other parties can implement an interface like
  \gls{textom} with functionality for phrases and clauses.

\section{Syntax: \textsc{nlcst}}\label{syntax}

To develop natural language tools in \gls{ecmascript}, an intermediate
  representation of natural language is useful.
Instead of each module (such as every stage in section
  \ref{sentiment-analysis} on page\,\pageref{sentiment-analysis})
  defining its own representation of text, using a single syntax leads to
  better interoperability, performance, and results.

The elements defined by \acrfull{nlcst} are based on the grammatical
  hierarchy, but by default do not expose all its constituents\footnote{The
      grammatical hierarchy of text is constituted by words, phrases,
      clauses, and sentences.
    \glspl{nlcst} only implements the sentence and word constituents
    by default, although clauses and phrases could be provided by
    implementations.}.
Additionally, \gls{nlcst} provides elements to cover other semantic units in
  natural language\footnote{Most
    notably punctuation, embedded content, and white space elements.}.

The definitions were influenced by other syntax trees specifications for
  manipulation on the web platform, such as \emph{\textsc{css}}, eponymous
  for the \acrshort{css} language \autocite{reworkcss/css-source-code} or the
  \emph{Mozilla JavaScript \textsc{ast}}, for \gls{ecmascript}
  \autocite{mozilla.org-spidermonkey-parser_api}.

Both widely used implementations, \emph{\textsc{css}} by Rework
  \autocite{reworkcss/rework-source-code}, and \emph{Mozilla JavaScript
  \textsc{ast}} by Esprima \autocite{ariya/esprima-source-code}, Acorn
  \autocite{marijnh/acorn-source-code}, and Escodegen
  \autocite{constellation/escodegen-source-code}.

\gls{nlcst} differs from both specifications in that it implements a
  \gls{cst}, where the others use an \gls{ast}.
A \gls{cst} is a one-to-one mapping of source (such as natural language)
  to result (a tree).
All information stored in the source is also available through the tree
  \autocite{thegreenplace.net-abstract-concrete-syntax-trees}.
This makes it easy for developers to save the output or pass it on to other
  libraries for further processing.
However, the information stored in \glspl{cst} is verbose, which might be
  difficult to work with.

\medskip \noindent See appendix\,\ref{appendix-nlcst} on
  page\,\pageref{appendix-nlcst} for a list of specified nodes of \gls{nlcst}.

\section{Parser: parse-latin}\label{parser-parse-latin}

To create a syntax tree according to \gls{nlcst} from natural language,
  this paper presents \emph{parse-latin} for Latin script based
  languages\footnote{Such
    as Old-English, Icelandic, French, or even scripts slightly similar, such
    as Cyrillic, Georgian, or Armenian.}.
Additionally, to prove the concept, two other libraries are presented,
  \emph{parse-english} and \emph{parse-dutch}. Both with \emph{parse-latin}
  as a basis, but providing better support for several language specific
  features, respectively for English and Dutch.

By using the \gls{cst} as described by \gls{nlcst} and the \emph{parse-latin}
  parser, the intermediate representation can be used by developers to
  create independent modules which may receive better results or
  performance over implementing their own parsing tools.

In essence, \emph{parse-latin} tokenises text into white space, word, and
  punctuation tokens.
\emph{parse-latin} starts out with a pretty simple definition, one that
  some other tokenisers also implement \autocite{treebank-tokenisation}:

\begin{enumerate}
\item A \emph{word} is one or more letter or number characters;
\item A \emph{white space} is one or more white space characters;
\item A \emph{punctuation} is one or more of anything else.
\end{enumerate}

\noindent Then, \emph{parse-latin} manipulates and merges those tokens into a
  syntax tree, adding sentences, paragraphs, and other nodes where needed.
Most of the intellect of the algorithm deals with sentence tokenisation
  (§\,\ref{sentence-tokenisation}, p.\,\pageref{sentence-tokenisation}).
This is done in similar fashion, but more intelligent, to \emph{Emphasis}
  \autocite{nytimes/emphasis-source-code}.

\begin{aenumerate}
\item\emph{Inter-word punctuation} --- Some punctuation marks are part of
  the word they occur in, such as the punctuation marks in ``non-profit'',
  ``she's'', ``G.I.'', ``11:00'', or ``N\slash A'';
\item\emph{Non-terminal full stops} --- Some full stops do not mark a
  sentence end, such as the full stops in ``1.'', ``e.g.'', or ``id.'';
\item\emph{Terminal punctuation} --- Although full stops, question marks,
  and exclamation marks (sometimes) end a sentence, that end might not occur
  directly after the mark, such as the punctuation marks after the full
  stop in ``.)'' or ``.'{}'';
\item\emph{Embedded content} --- Punctuation marks are sometimes used in
  non-standard ways, such as when a section or chapter delimiter is
  created with a line containing three asterisk marks (``* * *'').
\end{aenumerate}

\noindent See appendix\,\ref{appendix-parse-latin} on
page\,\pageref{appendix-parse-latin} for example output provided
by \emph{parse-latin}.

\subsection{parse-english}\label{parse-english}

\emph{parse-english} provides the same interface as \emph{parse-latin}, but
returns results better suited for English text.
Exceptions in the English language include:

\begin{aenumerate}
\item \emph{Unit abbreviations} --- ``tsp.'', ``tbsp.'', ``oz.'', ``ft.'',
  \&c.;
\item\emph{Time references} --- ``sec.'', ``min.'', ``tues.'', ``thu.'',
  ``feb.'', \&c.;
\item\emph{Business Abbreviations} --- ``Inc.'' and ``Ltd.''
\item\emph{Social titles} --- ``Mr.'', ``Mmes.'', ``Sr.'', \&c.;
\item\emph{Rank and academic titles} --- ``Dr.'', ``Gen.'', ``Prof.'',
  ``Pres.'', \&c.;
\item\emph{Geographical abbreviations} --- ``Ave.'', ``Blvd.'', ``Ft.'',
  ``Hwy.'', \&c.;
\item\emph{American state abbreviations} --- ``Ala.'', ``Minn.'', ``La.'',
  ``Tex.'', \&c.;
\item\emph{Canadian province abbreviations} --- ``Alta.'', ``Qué.'',
  ``Yuk.'', \&c.;
\item\emph{English county abbreviations} --- ``Beds.'', ``Leics.'',
  ``Shrops.'', \&c.;
\item\emph{Elision (omission of letters)} --- ``'n'\,'', ``'o'', ``'em'',
  ``'twas'', ``'80s'', \&c.
\end{aenumerate}

\subsection{parse-dutch}\label{parse-dutch}

\emph{parse-dutch} has, like \emph{parse-english}, the same interface as
\emph{parse-latin}, but returns results better suited for Dutch text.
Exceptions in the Dutch language include:

\begin{aenumerate}
\item\emph{Unit and time abbreviations} --- ``gr.'', ``sec.'', ``min.'', ``ma.'',
  ``vr.'', ``vrij.'', ``febr'', ``mrt'', \&c.;
\item\emph{Many other common abbreviations} --- ``Mr.'', ``Mv.'', ``Sr.'',
  ``Em.'', ``bijv.'', ``zgn.'', ``amb.'', \&c.;
\item\emph{Elision (omission of letters)} --- ``d'\,'', ``'n'', ``'ns'',
  ``'t'', ``'s'', ``'er'', ``'em'', ``'ie'', \&c.
\end{aenumerate}

\section{Object Model: Text\textsc{om}}\label{object-model}

To modify an \gls{nlcst} tree in \gls{ecmascript}, whether created by
  \emph{parse-latin}, \emph{parse-english}, \emph{parse-dutch}, or other
  parsers, this paper presents \gls{textom}.
\gls{textom} implements the nodes defined by \gls{nlcst},
  but provides an object-oriented style\footnote{Object-oriented
    programming is a style of programming, where classes, instances,
    attributes, and methods are important.} to manipulate these nodes.
\gls{textom} was designed in similarity to the \gls{dom}\footnote{See
    appendix\,\ref{appendix-dom} on page\,\pageref{appendix-dom} for more
    information on the \gls{dom}.},
  the mechanism used by browsers to expose \gls{html} through
  \gls{ecmascript} to developers.
Because of \glspl{textom} likeness to the \gls{dom}, \gls{textom} is
  easy to learn and familiar to the target audience.

\gls{textom} provides functionality for events (a mechanism for detecting
  changes), modification (inserting, removing, and replacing children
  into\slash from parents), and traversal (such as finding all words in a
  sentence).

\gls{nlcst} allows authors to extend the specification by defining their
  own units, such as creating phrase or clause nodes.
\gls{textom} allows for the same extension, and is built to work well
  with these ``unknown'' nodes.

\medskip \noindent See appendix\,\ref{appendix-textom} on
  page\,\pageref{appendix-textom} for the implementation details of
  \gls{textom}.

\section{Natural Language System:
  Retext}\label{natural-language-system-retext}

For natural language processing on the client side, this paper presents
  \emph{Retext}.
\emph{Retext} combines a parser, such as \emph{parse-latin} or
  \emph{parse-english}, with an object model: \gls{textom}.
Additionally, \emph{Retext} provides a minimalistic plug-in mechanism which
  enables developers to create and publish plug-ins for others to use, and in
  turn enables them to use others' plug-ins inside their projects.

\emph{Retext} provides a strong basis to use plug-ins to add simple natural
  language features to a website, but additionally provides functionality to
  extend this basis---create plug-ins, parsers, or other features---to
  create vast natural language systems.

\medskip \noindent See appendix\,\ref{appendix-retext} on
  page\,\pageref{appendix-retext} for a description of the interface
  provided by \emph{Retext} and example usage.
