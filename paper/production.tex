\chapter{Production}\label{production}

\section{Target Audience}\label{target-audience}

The audience that benefits the most from the proposal, are web developers.
Web developers are programmers who specialise in creating software that
  functions on the world wide web.
A group which enables machines to respond to humans.
They engage in client side development (building the interface between
  a human and a machine on the web), and sometimes also in server side
  development (building the interface between the client side and a
  server).

Typical areas of work consist of programming in \gls{ecmascript},
  marking up documents in \gls{html}, graphic design through \gls{css},
  creating a back end in Node.js, \gls{php}, or other platforms, contacting a
  \gls{mongodb}, \gls{mysql}, or other database, and more.

Additionally, many interdisciplinary skills, such as usability,
  accessibility, copywriting, information architecture, or optimisation,
  are also of concern to web developers.

\section{Use cases}\label{use-cases}

The use cases of the target audience, the web developer, in the field of
  \gls{nlp} are many.
Research for this paper found several use cases, although it is
  expected many more could be defined.
The tasks below are each categorised into broad, generic fields: analysation,
  manipulation, and creation.

\begin{aenumerate}
\item\label{list:use-case:1} The developer may intent to summarise natural
  text (mostly analysation, potentially also manipulation);
\item\label{list:use-case:2} The developer may intent to create natural
  language, e.g., displaying the number of unread messages: ``You have 1
  unread message,'' or ``You have 0 unread messages'' (creation);
\item\label{list:use-case:3} The developer may intent to recognise sentiment
  in text: is a \emph{tweet} positive, negative, or spam? (analysation);
\item\label{list:use-case:4} The developer may intent to replace so-called
  \emph{dumb} punctuation with \emph{smart} punctuation, such as dumb
  quotations with (``) or (''), three dots with an ellipsis (\ldots{}), or
  two hyphens with an en-dash (--) (manipulation);
\item\label{list:use-case:5} The developer may intent to count the number of
  certain grammatical units in a document, such as words, white space,
  punctuation, sentences, or paragraphs (analysation);
\item\label{list:use-case:6} The developer may intent to recognise the
  language in which a document is written (analysation);
\item\label{list:use-case:7} The developer may intent to find words in a
  document based on a search term, with regards for the lemma (or stem)
  and\slash or phonetics (so that a search for ``smit'' also returns similar
  words, such as ``Schmidt'' or ``Smith'') (analysation and manipulation).
\end{aenumerate}

\noindent \gls{nlp} is a large field with many challenges, but not every
  challenge in the field is of interest to the web developer.
Foremost, the more academic areas of \gls{nlp}, such as speech recognition,
  optical character recognition, text to speech transformation, translation,
  and machine learning, do not fit well with the goals of web developers.

\section{Requirements}\label{requirements}

The proposal must enable the target audience to reach the in the previous
  section defined use cases.
In addition, the proposal should meet several other requirements to better
  suit the wishes of the target audience.

\subsection{Open Source}\label{open-source}

To reach the target audience and validate its usability, the proposal
  should be open source.
All code should be licensed under \acrshort{mit}, a license which
  provides rights for others to use, copy, modify, merge, publish,
  distribute, sublicense, and\slash or sell copies of the code it covers.

In addition, the software should be developed under the all-seeing eye of
  the community: on GitHub.
GitHub is a hosted version control\footnote{Version
    control services manage revisions to documents, popularly used for
    controlling and tracking changes in software.} service with social
  networking features.
On GitHub, web developers follow their peers to track what they are
  working on, watch their favourite projects to get notified of changes,
  and raise issues and request features.

\subsection{Performance}\label{performance}

The proposal should execute at high \emph{performance}.
Performance includes the software having a small file size to reach the
  client over the network with the highest possible speed, but most
  importantly that the execution of code should run efficiently and at high
  speeds.

\subsection{Testing}\label{testing}

\emph{Testing} should have high priority in the proposal.
Testing, in software development, refers to validating if software does what
  it is supposed to do, and can be divided into several subgroups:

\begin{aenumerate}
\item\emph{Unit testing} --- Validation of each specific section of code;
\item\emph{Integration testing} --- Validation of how programs work together;
\item\emph{System testing} --- Validation of if the system meets its
  requirements;
\item\emph{Acceptance testing} --- Validation of the end product.
\end{aenumerate}

\noindent Great care should be given to develop an adequate test suite with
  full \emph{coverage} for every program.
Coverage, in software development, is a term used to describe the amount of
  code tested by the test suite.
Full coverage means every part of the code is reached by the tests.

Unit test run through Mocha \autocite{visionmedia/mocha-source-code},
  coverage is detected by Istanbul
  \autocite{gotwarlost/istanbul-source-code}.

\subsection{Code quality}\label{code-quality}

\emph{Code quality}---how useful and readable for both humans and machines
  the software is---should be vital. For humans, the code should be
  consistent and clear. For computers, the code should be free of bugs and
  other suspicious code.

\subsubsection{Suspicious Code and Bugs}\label{suspicious-code-and-bugs}

To detect bugs and suspicious code in the software, \emph{Eslint}
  is used \autocite{eslint/eslint-source-code}.
\emph{Linting}, in computer programming, is a term used to describe static
  code analysis to detect syntactic discrepancies without running the code.
\emph{Eslint} is used because it provides a solid set of basic rules and
  enables developers to create custom rules.

\subsubsection{Style}\label{style}

To enforce a consistent code style---to create readable software for
  humans---\acrshort{jscs} is used \autocite{mdevils/node-jscs-source-code}.
\acrshort{jscs} provides rules for (dis)allowing certain code patterns,
  such as white space at the end of a line or camel cased variable names,
  or enforcing a maximum line length.
\acrshort{jscs} was chosen because it, like \emph{Eslint}, provides a strong
  basic set of rules.
The rules chosen for the proposal were set strict to enforce all code to be
  written in the same way.

\subsubsection{Commenting}\label{commenting}

Even when code is bug free, uses no confusing short-cuts, and adheres to a
  strict style, it might still be hard to understand for humans.
\emph{Commenting} code---describing what a program does and why it
  accomplishes it this way---is important.
However, commenting can also be too verbose, such as when the code is
  duplicated in natural language.

\gls{jsdoc} \autocite{google.com-clojure-compiler-jsdoc} is a markup language
  for \gls{ecmascript} that allows developers to embed documentation---using
  comments---in source code.
Several tools can later extract this information and expose it independent
  from the original code.
``Tricky'' code should be annotated inside the software with comments to help
  readers understand why certain decisions were made.

\subsection{Automation}\label{automation}

When suspicious, ambiguous, or buggy code is introduced in the software, the
  error should be automatically detected.
Sometimes, deployment should be prevented.
Automated \gls{ci} environments to enforce error detection should be used.
To detect complex, duplicate, or bug prone code, Code Climate is used
  \autocite{codeclimate.com}.
To validate all tests passed before deploying the software, Travis is used
  \autocite{travis-ci.org}.

\subsection{\textsc{api} Design}\label{design-1}

Quality interface design should have high priority for the proposal.
A good \gls{api}, according to Joshua Bloch
  \autocite*{bloch-joshua-how-design-good-api-why-matters}, has the following
  characteristics:

\begin{enumerate}
\item Easy to learn;
\item Easy to use;
\item Hard to misuse;
\item Easy to read;
\item Easy to maintain;
\item Easy to extend;
\item Meeting its requirements;
\item Appropriate for the target audience.
\end{enumerate}

\noindent In essence equal, but worded differently, are the characteristics
  of good \gls{api} design according to the Qt Project
  \autocite*{qt-project.org-api-design-principles}:

\begin{enumerate}
\item Be minimal;
\item Be complete;
\item Have clear and simple semantics;
\item Be intuitive;
\item Be easy to memorise;
\item Lead to readable code.
\end{enumerate}

\noindent The proposal should take these characteristics, and the in their
  sources given examples, into account.

\subsection{Installation}\label{installation}

Simple access to the software for the target audience, both on the client
  side and on the server side, should be given high priority.
On the client side, many package managers exist, the most
  popular being Bower and Component\footnote{Popularity here is simply
    defined as having the highest number of search results on Google.}.
For Node.js (on the server side), \gls{npm} is the most popular.
To reach the target audience, besides making the source available on GitHub,
  all popular package managers, \gls{npm}, Bower, and Component, are used.
