%*******************************************************
% Abstract
%*******************************************************
% to have the introduction a bit from the rest in the toc
\refstepcounter{dummy}
\addtocontents{toc}{\protect\vspace{\beforebibskip}}
\addcontentsline{toc}{chapter}{\tocEntry{Introduction}}

\begingroup
\let\clearpage\relax
\let\cleardoublepage\relax
\let\cleardoublepage\relax

\chapter*{Introduction}

\Gls{nlp}, a field of computer science, artificial intelligence, and
  linguistics concerned with the interaction between computers and human
  languages, is becoming more important in society.
For example, search engines provide answers before being questioned,
  intelligence agencies detect threats of violence in text messages, and
  e-mail applications know if you forgot to include an attachment.

Despite increased interest, web developers trying to solve \gls{nlp} problems
  reinvent the wheel over and over.
There are tools, especially for other platforms---such as in Python
  \autocite{nltk-source} and Java \autocite{opennlp-source}---but they either
  take a too na\"ive approach\footnote{Such as ignoring white space
    \autocite{loadfive/knwl-source-code}, implementing a na\"ive
    definition of ``words'' \autocite{nhunzaker/speakeasy-source-code},
    or by using an inadequate algorithm to detect sentences
    \autocite[][]{nytimes/emphasis-source-code}.}, or try to do
  everything\footnote{Although a do-all library \autocite[such
  as][]{NaturalNode/natural-source-code} works well on server side
  platforms, it fares less well on the web, where modularity and moderation
  are in order.}.
What is missing is a standard representation of the grammatical hierarchy
  of text and a standard for multipurpose analysis of natural language.

My initial interest in natural language was sparked by typography, when I
  felt the need to create a typographically beautiful website, somewhere in
  the summer of 2013.
I felt a craving to apply the tried-and-true practices of typography found on
  paper, to the web.
I was inspired by how these practices were available on other platforms,
  on \TeX{} or \LaTeX, with tools such as \emph{microtype} \autocite{microtype},
  and the \emph{ClassicThesis} theme \autocite{classicthesis} based on
  \emph{The Elements of Typographic Style}
  \autocite{bringhurst-element-typographic-style}.

My interest for fixing typography on the web was piqued.
Thus, I began work on \emph{MicroType.js}, an unpublished library, to
  enable several graphic and typographic practices on the web.
Examples include automatic initials, ligatures, optical margin alignment,
  acronym recognition, smart punctuation, automatic hyphenation, character
  transpositions, and more.
The possibilities were vast, but I noticed how the underlying parser and data
  representation were incomplete.
How words, white space, punctuation, and sentences were defined, was not
  good enough.
The website never came into existence, but during this thesis, I could
  finally fix this problem.
While working on this thesis, the specification, the product, I developed a
  well thought out and substantiated solution.

\emph{Retext}---the implementation introduced in this thesis---and other
  projects in the \emph{Retext} family are a new approach to the syntax of
  natural text.
Together they form an extensible system for multipurpose analysis of natural
  language in \gls{ecmascript}.

To reach this goal, I have organised my paper into six\,main chapters.
In the first\,chapter I define the scope of this paper and review current
  implementations, what they lack and where they excel.
In the second\,chapter I state a research objective and draft research
  questions.
The third\,chapter defines conditions for such a proposal, where I touch
  upon the target audience, use cases, and requirements.
In the fourth\,chapter, I propose a better implementation and show its
  architectural design.

The fifth\,chapter describes the steps taken to validate the proposal.
I conclude with a sixth\,chapter that offers information on expanding the
  proposal.

\endgroup
