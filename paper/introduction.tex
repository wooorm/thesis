%*******************************************************
% Abstract
%*******************************************************
% to have the introduction a bit from the rest in the toc
\refstepcounter{dummy}
\addtocontents{toc}{\protect\vspace{\beforebibskip}}
\addcontentsline{toc}{chapter}{\tocEntry{Introduction}}

\begingroup
\let\clearpage\relax
\let\cleardoublepage\relax
\let\cleardoublepage\relax

\chapter*{Introduction}

\Gls{nlp}, a field of computer science, artificial intelligence, and
  linguistics concerned with the interaction between computers and human
  languages, is becoming more important in society.
For example, search engines provide answers before being questioned,
  intelligence agencies detects threats of violence in text messages, and
  e-mail applications know if you forgot to include an attachment.

Despite increased interest, web developers trying to solve \gls{nlp} problems
reinvent the wheel over and over again. There are tools,
especially for other platforms---such as in Python
\autocite{nltk-source} and Java \autocite{opennlp-source}---but they either
take a too naive approach\footnote{Such as ignoring white space
  \autocite{loadfive/knwl-source-code}, implementing a naive
  definition of ``words'' \autocite{nhunzaker/speakeasy-source-code},
  or by using an inadequate algorithm to detect sentences
  \autocite[][]{nytimes/emphasis-source-code}.}, or try to do everything out
of the box\footnote{Although a do-all library works well on server-side
  platforms, it fares less well on the web \autocite[such
  as][]{NaturalNode/natural-source-code}, where modularity and moderation
  are in order.}. What is missing is a standard representation of the
grammatical hierarchy of text and a standard for multipurpose analysis of
natural language.

My initial motivation for natural language was sparked by typography, when I
  felt the need to create a typographically beautiful blog, somewhere in the
  summer of 2013.
I felt a craving to apply the tried-and-true practices of typography found on
  paper, to the web.
I was inspired by how these practices were available on other platforms,
  on \TeX or \LaTeX, with tools such as \emph{microtype} \autocite{microtype},
  and the \emph{ClassicThesis} theme \autocite{classicthesis} based on
  \emph{The Elements of Typographic Style}
  \autocite{bringhurst-element-typographic-style}.

My interest for fixing typography on the web was piqued.
Thus, I began work on \emph{MicroType.js}, an unpublished library, to
  facilitate several graphic and typographic practices on the web.
Examples include automatic initials, ligatures, optical margin alignment,
  acronym recognition, smart punctuation, automatic hyphenation, character
  transpositions, and more.
The possibilities were vast, but I noticed how the underlying parser and data
  representation, the definition of words, white space, punctuation, and
  sentences, was not good enough.
The blog never came into existence, but during this thesis, I had the chance
  to finally fix this problem.
While working on this thesis, the specification, the product, I developed a
  well thought out and substantiated solution.

Retext---the implementation introduced in this thesis---and other projects
in the Retext family are a new approach to the syntax of natural text.
Together they form an extensible system for multipurpose analysis of natural
language in \gls{ecmascript}.

To achieve this goal, I have organised my paper into five main chapters. In
the first chapter I define the scope of this paper and review current
implementations, what they lack and where they excel. In the second
chapter, I propose a better implementation and show its architectural
design. In the third chapter, I define conditions for the production of
such a proposal, where I touch upon the target audience, use cases, and
requirements.

I end the paper with a fourth chapter which describes the steps taken to
validate the proposal. I conclude with a fifth chapter that offers
information on expanding the proposal.

Before I can begin the examination of a proposal, I need to provide a
context for \gls{nlp}.

\endgroup
