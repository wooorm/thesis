%*******************************************************
% Abstract
%*******************************************************
\pdfbookmark[1]{Introduction}{introduction}

\begingroup
\let\clearpage\relax
\let\cleardoublepage\relax
\let\cleardoublepage\relax

\chapter*{Introduction}

\Gls{nlp}, a field of computer science, artificial intelligence, and
linguistics concerned with the interactions between computers and human
languages \autocite[according to WikiPedia, see
][]{wikipedia-natural-language-processing}, is becoming increasingly
important in society. For example, search engines try to answer before a
question is given, the NSA detects terrorist-toned motifs in text
messages, and e-mail applications know if a user forgot to attach an
attachment.

However, to date, web developers trying to solve \gls{nlp} problems
reinvent the wheel over and over again. There are certainly tools,
especially for other platforms. such as Python
\autocite{nltk-source} and Java \autocite{opennlp-source},
trying to solve this, but they either take a too naive
approach\footnote{For example, ignoring white space
  \autocite[See][]{loadfive/knwl-source-code}, punctuation, or
  implementing a naive definition of ``words'' \autocite[such
  as][]{nhunzaker/speakeasy-source-code}, or by deploying a less than
  adequate algorithm to detect sentences \autocite[such
  as][]{nytimes/emphasis-source-code}.}, or try to do everything out of
the box\footnote{Although a do-all library works well on server-side
  platforms, it fares less well on the web \autocite[such
  as][]{NaturalNode/natural-source-code}, where modularity and frugality
  are in order.}. What is missing is a standard for multipurpose natural
language analysis---a standard representation of the grammatical
hierarchy of text.

For over a year I too have camped with the previously mentioned problem.
I have tried to solve it, to no avail, on numerous occasions. During the
creation of this thesis I developed a well thought out and substantiated
solution, which solves the previously mentioned problems.

The in this thesis introduced implementation Retext (and other projects
in the Retext family) are a new approach to the syntax of natural text.
My goal in this paper is to design an extensible system for multipurpose
analysis of language.

To achieve this goal, I have organised my paper into five main chapters.
In the first chapter I define the scope of this paper and review current
implementations, what they lack and where they excel.
In the second section, I propose a better implementation and show its
architectural design. Then, I continue in the third chapter with the
production of such a proposal: target audience, use cases, and
requirements.

I end my paper with a fourth chapter that describe the steps I took to validate
my proposal and conclude with a fifth chapter that offers information on
expanding the proposal.

Before I can begin the examination of a proposal, however, I need to
provide a context for \gls{nlp}.

\endgroup

