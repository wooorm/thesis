%*******************************************************
% Abstract
%*******************************************************
\pdfbookmark[1]{Introduction}{introduction}

\begingroup
\let\clearpage\relax
\let\cleardoublepage\relax
\let\cleardoublepage\relax

\chapter*{Introduction}

\Gls{nlp}, a field of computer science, artificial intelligence, and
linguistics concerned with the interaction between computers and human
languages \autocite[according to WikiPedia, see
][]{wikipedia-natural-language-processing}, is becoming more
important in society. For example, search engines provide answers before
being questioned, intelligence agencies detects threats of violence in text
messages, and e-mail applications know if the user forgot to include an
attachment.

Despite increased interest, web developers trying to solve \gls{nlp} problems
reinvent the wheel over and over again. There are tools,
especially for other platforms---such as in Python
\autocite{nltk-source} and Java \autocite{opennlp-source}---but they either
take a too naive approach\footnote{Such as ignoring white space
  \autocite{loadfive/knwl-source-code}, implementing a naive
  definition of ``words'' \autocite{nhunzaker/speakeasy-source-code},
  or by using an inadequate algorithm to detect sentences
  \autocite[][]{nytimes/emphasis-source-code}.}, or try to do everything out
of the box\footnote{Although a do-all library works well on server-side
  platforms, it fares less well on the web \autocite[such
  as][]{NaturalNode/natural-source-code}, where modularity and moderation
  are in order.}. What is missing is a standard representation of the
grammatical hierarchy of text and a standard for multipurpose analysis of
natural language.

For over a year I too camped with this problem. On many occasions I
tried solving it, to no avail. While working on this thesis, the
specification, the product, I developed a well thought out and substantiated
solution.

Retext---the implementation introduced in this thesis---and other projects
in the Retext family are a new approach to the syntax of natural text.
Together they form an extensible system for multipurpose analysis of natural
language in \gls{ecmascript}.

To achieve this goal, I have organised my paper into five main chapters. In
the first chapter I define the scope of this paper and review current
implementations, what they lack and where they excel. In the second
chapter, I propose a better implementation and show its architectural
design. In the third chapter, I define conditions for the production of
such a proposal, where I touch upon the target audience, use cases, and
requirements.

I end the paper with a fourth chapter which describes the steps taken to
validate the proposal. I conclude with a fifth chapter that offers
information on expanding the proposal.

Before I can begin the examination of a proposal, I need to provide a
context for \gls{nlp}.

\endgroup
