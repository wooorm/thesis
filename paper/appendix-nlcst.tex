%********************************************************************
% Appendix NLCST
%*******************************************************

\chapter{NLCST definition}\label{appendix-nlcst}

\section*{Node}\label{node}

Node represents any unit in the \gls{nlcst} hierarchy.

\begin{lstlisting}
interface Node {
    type: string;
}
\end{lstlisting}

\section*{Parent}\label{parent}

Parent (Node) represents a unit in the \gls{nlcst} hierarchy which can have
  zero or more children.

\begin{lstlisting}
interface Parent <: Node {
    children: [];
}
\end{lstlisting}

\section*{Text}\label{text}

Text (Node) represents a unit in the \gls{nlcst} hierarchy which has a
  value.

\begin{lstlisting}
interface Text <: Node {
    value: string | null;
    location: Location | null;
}
\end{lstlisting}

\section*{Location}\label{location}

Location represents the node's location in the source input.

\begin{lstlisting}
interface Location {
    start: Position;
    end: Position;
}
\end{lstlisting}

\section*{Position}\label{position}

Position represents a position in the source input.

\begin{lstlisting}
interface Position {
    line: uint32 >= 1;
    column: uint32 >= 1;
}
\end{lstlisting}

\section*{RootNode}\label{rootnode}

Root (Parent) represents the document.

\begin{lstlisting}
interface RootNode < Parent {
    type: "RootNode";
}
\end{lstlisting}

\section*{ParagraphNode}\label{paragraphnode}

Paragraph (Parent) represents a self-contained unit of a discourse in
writing dealing with a particular point or idea.

\begin{lstlisting}
interface ParagraphNode < Parent {
    type: "ParagraphNode";
}
\end{lstlisting}

\section*{SentenceNode}\label{sentencenode}

Sentence (Parent) represents a grouping of grammatically linked words,
that in principle tells a complete thought (although it may make little
sense taken in isolation out of context).

\begin{lstlisting}
interface SentenceNode < Parent {
    type: "SentenceNode";
}
\end{lstlisting}

\section*{WordNode}\label{wordnode}

Word (Parent) represents the smallest element that may be uttered in
isolation with semantic or pragmatic content.

\begin{lstlisting}
interface WordNode < Parent {
    type: "WordNode";
}
\end{lstlisting}

\section*{PunctuationNode}\label{punctuationnode}

Punctuation (Parent) represents typographical devices which aids the
understanding and correct reading of other grammatical units.

\begin{lstlisting}
interface PunctuationNode < Parent {
    type: "PunctuationNode";
}
\end{lstlisting}

\section*{WhiteSpaceNode}\label{whitespacenode}

White Space (Punctuation) represents typographical devices devoid of
content, separating other grammatical units.

\begin{lstlisting}
interface WhiteSpaceNode < PunctuationNode {
    type: "WhiteSpaceNode";
}
\end{lstlisting}

\section*{SourceNode}\label{sourcenode}

Source (Text) represents an external (ungrammatical) value embedded
into a grammatical unit, for example a hyperlink or an emoticon.

\begin{lstlisting}
interface SourceNode < Text {
    type: "SourceNode";
}
\end{lstlisting}

\section*{TextNode}\label{textnode}

Text (Text) represents actual content in an \gls{nlcst} document: one\,or more
characters.

\begin{lstlisting}
interface TextNode < Text {
    type: "TextNode";
}
\end{lstlisting}
